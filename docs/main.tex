%%%%%%%%%%%%%%%%%%%%%%%%%%%%%%%%%%%%%%%%%
% Wenneker Assignment
% LaTeX Template
% Version 2.0 (12/1/2019)
%
% This template originates from:
% http://www.LaTeXTemplates.com
%
% Authors:
% Vel (vel@LaTeXTemplates.com)
%
% License:
% CC BY-NC-SA 3.0 (http://creativecommons.org/licenses/by-nc-sa/3.0/)
% 
%%%%%%%%%%%%%%%%%%%%%%%%%%%%%%%%%%%%%%%%%

%----------------------------------------------------------------------------------------
%	PACKAGES AND OTHER DOCUMENT CONFIGURATIONS
%----------------------------------------------------------------------------------------

\documentclass[11pt]{scrartcl} % Font size
%%%%%%%%%%%%%%%%%%%%%%%%%%%%%%%%%%%%%%%%%
% Wenneker Assignment
% Structure Specification File
% Version 2.0 (12/1/2019)
%
% This template originates from:
% http://www.LaTeXTemplates.com
%
% Authors:
% Vel (vel@LaTeXTemplates.com)
% Frits Wenneker
%
% License:
% CC BY-NC-SA 3.0 (http://creativecommons.org/licenses/by-nc-sa/3.0/)
% 
%%%%%%%%%%%%%%%%%%%%%%%%%%%%%%%%%%%%%%%%%

%----------------------------------------------------------------------------------------
%	PACKAGES AND OTHER DOCUMENT CONFIGURATIONS
%----------------------------------------------------------------------------------------

\usepackage{amsmath, amsfonts, amsthm} % Math packages
\usepackage{algorithm}
\usepackage{algpseudocode}
\usepackage{listings} % Code listings, with syntax highlighting

\usepackage[english]{babel} % English language hyphenation
\usepackage{amsmath}
\DeclareMathOperator*{\argmax}{arg\,max}
\usepackage{graphicx} % Required for inserting images
\graphicspath{{Figures/}{./}} % Specifies where to look for included images (trailing slash required)

\usepackage{booktabs} % Required for better horizontal rules in tables

\numberwithin{equation}{section} % Number equations within sections (i.e. 1.1, 1.2, 2.1, 2.2 instead of 1, 2, 3, 4)
\numberwithin{figure}{section} % Number figures within sections (i.e. 1.1, 1.2, 2.1, 2.2 instead of 1, 2, 3, 4)
\numberwithin{table}{section} % Number tables within sections (i.e. 1.1, 1.2, 2.1, 2.2 instead of 1, 2, 3, 4)

\setlength\parindent{0pt} % Removes all indentation from paragraphs

\usepackage{enumitem} % Required for list customisation
\setlist{noitemsep} % No spacing between list items

%----------------------------------------------------------------------------------------
%	DOCUMENT MARGINS
%----------------------------------------------------------------------------------------

\usepackage{geometry} % Required for adjusting page dimensions and margins

\geometry{
	paper=a4paper, % Paper size, change to letterpaper for US letter size
	top=2.5cm, % Top margin
	bottom=3cm, % Bottom margin
	left=3cm, % Left margin
	right=3cm, % Right margin
	headheight=0.75cm, % Header height
	footskip=1.5cm, % Space from the bottom margin to the baseline of the footer
	headsep=0.75cm, % Space from the top margin to the baseline of the header
	%showframe, % Uncomment to show how the type block is set on the page
}

%----------------------------------------------------------------------------------------
%	FONTS
%----------------------------------------------------------------------------------------

\usepackage[utf8]{inputenc} % Required for inputting international characters
\usepackage[T1]{fontenc} % Use 8-bit encoding

\usepackage{fourier} % Use the Adobe Utopia font for the document

%----------------------------------------------------------------------------------------
%	SECTION TITLES
%----------------------------------------------------------------------------------------

\usepackage{sectsty} % Allows customising section commands

\sectionfont{\vspace{6pt}\centering\normalfont\scshape} % \section{} styling
\subsectionfont{\normalfont\bfseries} % \subsection{} styling
\subsubsectionfont{\normalfont\itshape} % \subsubsection{} styling
\paragraphfont{\normalfont\scshape} % \paragraph{} styling

%----------------------------------------------------------------------------------------
%	HEADERS AND FOOTERS
%----------------------------------------------------------------------------------------

\usepackage{scrlayer-scrpage} % Required for customising headers and footers

\ohead*{} % Right header
\ihead*{} % Left header
\chead*{} % Centre header

\ofoot*{} % Right footer
\ifoot*{} % Left footer
\cfoot*{\pagemark} % Centre footer
 % Include the file specifying the document structure and custom commands

%----------------------------------------------------------------------------------------
%	TITLE SECTION
%----------------------------------------------------------------------------------------

\title{	
	\normalfont\normalsize
	\rule{\linewidth}{0.5pt}\\ % Thin top horizontal rule
	\vspace{10pt} % Whitespace
	{\huge Rozwiązywanie Sudoku przy pomocy algorytmu genetycznego oraz algorytmu ACO}\\ % The assignment title
	\vspace{12pt} % Whitespace
	\rule{\linewidth}{2pt}\\ % Thick bottom horizontal rule
}

\author{\LARGE Bartosz Cywiński, Łukasz Staniszewski} % Your name

\date{} % Today's date (\today) or a custom date

\begin{document}

\maketitle % Print the title
%----------------------------------------------------------------------------------------
%	FIGURE EXAMPLE
%----------------------------------------------------------------------------------------

\section{Opis problemu}
Podstawowa wersja Sudoku składa się z dwuwymiarowej planszy 9x9, podzielonej na 9 pól 3x3. Planszę Sudoku da się jednak rozszerzać do większych rozmiarów. Plansza na początku gry częściowo uzupełniona jest liczbami. Początkowy rozkład liczb ma duże znaczenie dla dalszego przebiegu gry – jest on jedynym czynnikiem decydującym o trudności danej wersji Sudoku. Celem rozwiązania łamigłówki jest takie ułożenie cyfr od 1 do 9, aby w każdym rzędzie, każdej kolumnie i każdym polu 3x3 każda cyfra występowała dokładnie jeden raz. Zostało udowodnione, że problem rozwiązania Sudoku jest problemem NP-trudnym, dlatego też jest on dobrym testem efektywności algorytmów.

%------------------------------------------------
\section{Opis rozwiązania przy pomocy algorytmu ACO}

\subsection{Pseudokod i opis algorytmu}
\begin{algorithm}
\caption{Algorytm ACS do rozwiązywania Sudoku}\label{alg:cap}
\begin{algorithmic}[1]
\State \texttt{wczytaj planszę Sudoku}
\For{\texttt{komórka z wybraną wartością}}
	\State \texttt{aktualizuj możliwe do przyjęcia wartości przez sąsiadów tej komórki}
\EndFor
\State \texttt{zainicjuj tablicę feromonów}
\While {\texttt{!stop}}
	\State \texttt{daj każdej mrówce kopię planszy Sudoku}
	\For{\texttt{liczba komórek}}
		\For{\texttt{liczba mrówek}}
			\If{\texttt{komórka nie ma wybranej wartości}}	
				\State \texttt{wybierz wartość dla komórki}
				\State \texttt{aktualizuj ograniczenia}
				\State \texttt{aktualizuj lokalny feromon}
			\EndIf
		\EndFor
	\EndFor
	\State \texttt{znajdź najlepszą mrówkę}
	\State \texttt{aktualizuj tablicę feromonów}
	\State \texttt{parowanie najlepszej wartości}
\EndWhile

\end{algorithmic}
\end{algorithm}

W zaproponowanym rozwiązaniu użyty został algorytm Ant Colony System (ACS), będący ulepszoną wersją klasycznego algorytmu Ant System (AS). Jego cechą charakterystyczną jest zastosowanie lokal\-nych aktualizacji feromonów przez każdą mrówkę oraz to, że tylko najlepsza mrówka aktualizuje globalne wartości tablicy feromonów. Ponadto po każdym wybraniu nowego elementu rozwiązania (wpisaniu cyfry w wybraną komórkę planszy) aktualizowane zostają wszystkie wartości możliwe do przyjęcia przez komórki planszy oraz spełniające zasady łamigłówki, ponieważ na skutek wybrania elementu rozwiązania mogły się one zmienić.\\


\textit{Linie 2-4:} Każda komórka o indeksie $i$ na początku algorytmu posiada taki sam zbiór możliwych wartości do przyjęcia ($\mathbb{V}_i=\{1,...,9\}$). Aktualizacje ograniczeń komórek z wybraną wartością polegają na:
\begin{itemize}
	\item Eliminacji z $\mathbb{V}_i$ wartości wybranych przez sąsiadów komórki $i$, gdzie sąsiedzi to wszystkie komór\-ki w tym samym rzędzie, tej samej kolumnie oraz tym samym polu 3x3, 
	\item Jeśli $|\mathbb{V}_i| = 1$, to przypisz wartość do komórki. 
\end{itemize}


\textit{Linia 5:} Dla Sudoku o wymiarze $d$ definiujemy dwuwymiarową globalną tablicę feromonów $\tau$, w której każdy element jest oznaczony jako $\tau_{i}^{k}$, gdzie $i$ to indeks komórki planszy $(0\leq{i}\leq{d^2-1})$, a $k$ to możliwa wartość do przyjęcia przez komórkę $(k\in{[1,d]})$. Każdy element tablicy feromonów inicjowany jest wartością $\tau_{0}=\frac{1}{c}$, gdzie $c=d^2$ jest całkowitą liczbą komórek planszy. W tablicy trzymane są wartości feromonów każdej możliwej do przyjęcia wartości w każdej komórce planszy.\\

\textit{Linie 8-10:} Zbiór $m$ mrówek buduje częściowe rozwiązanie $s^p$ ze skończonego zbioru możliwych elementów rozwiązania $\textbf{C} = \{c_{i}^{k}\}$, $i=0,...,d^2 -1,$ $k\in{[1,d]}$, gdzie $i$ to indeks komórki planszy, a $k$ to możliwa cyfra do wpisania w komórkę. Każda mrówka zaczyna od losowo wybranej komórki planszy Sudoku. Przechodzi ona po każdej możliwej komórce na planszy. Za każdym razem, kiedy mrówka natrafia na komórkę, która nie ma przypisanej wartości, podejmuje ona decyzję jaka to powinna być wartość i ją przypisuje - do $s^p$ dodawany jest element rozwiązania ze zbioru $\textbf{N}(s^p)\subseteq{\textbf{C}}$, który jest zdefiniowany jako zbiór elementów, które mogą być dodane do częściowego rozwiązania $s^p$ bez naruszenia założonych ograniczeń.\\

\textit{Linia 11:} Wybór wartości dla komórki $i$ dokonywany jest za pomocą stochastycznego mechanizmu uwzględniającego wartości w tablicy feromonów. Zdefiniujmy zbiór $\mathbb{V}_i$, jako zbiór możliwych do przyjęcia wartości przez komórkę $i$. Istnieją do wyboru dwie metody:
\begin{itemize}
	\item Selekcja zachłanna, gdzie wybieramy element ze zbioru $\mathbb{V}_i$ z najwyższą wartością w tablicy feromonów, 
	\item Selekcja ruletkowa, gdzie prawdopodobieństwa są proporcjonalne do wartości w tablicy feromonów. 
\end{itemize}
Prawdopodobieństwa wybierane są na podstawie parametru $ q_0\in{[0,1]} $. Kolejny element rozwiązania $s$ jest zatem wybierany według wzoru:
\begin{equation}
s = \begin{cases}
\argmax_{{k\in\mathbb{V}_i}}{\tau_i^k}, & \text{jeśli $q\leq{q_0}$},\\
$R$, & \text{w pozostałych przypadkach},
\end{cases}
\end{equation}
gdzie wartość $ q \in{[0,1]}$ losowana jest z rozkładem jednostajnym, natomiast $R$ jest określona jako:
\begin{equation}
p_{i}^{k} =
\dfrac{\tau_{i}^{k}}{\sum_{j\in{\mathbb{V}_i}}{\tau_i^j}}, k\in\mathbb{V}_i
\end{equation}
gdzie $p_{i}^{k}$ to prawdopodobieństwo wyboru $k$ z $\mathbb{V}_i$ na miejscu o indeksie $i$.\\

\textit{Linia 13:} Aktualizowanie lokalnego feromonu wspomaga eksplorację przestrzeni przeszukiwań. Za każdym razem kiedy wartość $s$ zostaje przypisana do komórki w planszy o indeksie $i$, aktualizacja następuje według formuły:
\begin{equation}
\tau_i^s = (1 - \varphi)\tau_i^s + \varphi\tau_0,
\end{equation}
gdzie $\varphi \in (0,1]$ to parametr rozpadu feromonu, który jest zazwyczaj mały, a $\tau_0$ jest początkową wartością feromonu.\\


Celem aktualizacji tablicy feromonów jest zwiększenie wartości feromonów związanych z dobrymi rozwiązaniami i zmniejszenie wartości związanych z gorszymi rozwiązaniami.\\

\textit{Linie 17-18:} Aby znaleźć najlepsze rozwiązanie, należy znaleźć najlepszą mrówkę w iteracji, czyli taką, która uzupełniła najwięcej komórek w planszy Sudoku w danej iteracji algorytmu. Aktualizacja tablicy feromonów następuje według wzoru:
\begin{equation}
\tau_i^s = \begin{cases}
(1-\rho)\tau_i^s+\rho\Delta\tau_i^s, & \text{jeśli najlepsze rozwiązanie w iteracji jest najlepsze globalnie},\\
\tau_i^s, & \text{w pozostałych przypadkach,}
\end{cases}
\end{equation}
gdzie $\rho \in [0,1]$ to parametr parowania feromonu, a $\Delta\tau_i^s$ można przyjąć jako:
\begin{equation}
\Delta\tau_i^s = \dfrac{c}{c-f_{best}},
\end{equation}
gdzie $f_{best}$ to liczba komórek uzupełnionych przez najlepszą mrówkę, a $c$ to całkowita liczba komórek na planszy.\\

\textit{Linia 19:} Na koniec stosowane jest parowanie najlepszej wartości, które zapobiega przedwczesnej zbieżności algorytmu:
\begin{equation}
\Delta\tau_{best} = \Delta\tau_{best}(1-\rho_{best}),
\end{equation}
gdzie $\rho_{best}$ to parametr parowania feromonu najlepszej wartości.\\
\section{Opis rozwiązania przy pomocy algorytmu genetycznego}


\subsection{Opis algorytmu}

Przed zdefiniowaniem samego zastosowania algorytmu do problemu, dobrze jest określić części na jakie składa się Algorytm Genetyczny. W typowym takim algorytmie zaczyna się od inicjalizacji populacji startowej - jest to populacja, z którą zaczynamy. Następnie, algorytm opiera się na pętli, która trwa do momentu spełnienia warunku stopu. W pętli tej pierwszym krokiem jest ocenienie całej aktualnej populacji przy użyciu funkcji dopasowania i posortowanie populacji rosnąco względem tego wskaźnika. Następnie wykonywana jest selekcja, polegająca na wybraniu z aktualnej populacji tych osobników, którzy mają przejść do kolejnej populacji, pozostałych się odrzuca. Kolejno, do momentu aż rozmiar nowej populacji nie pokryje się z rozmiarem poprzedniej, wybierani są z wyselekcjonowanej części dwaj rodzice ze zwracaniem, na nich wykonywane jest krzyżowanie (gdzie mieszamy informację o jednym rodzicu z informacją o drugim rodzicu), w wyniku czego otrzymujemy dwoje dzieci, każdy zawierający inne części swoich rodziców. Na takich dzieciach wykonywana jest mutacja, polegająca na losowej zamianie elementów w chromosomie.\\

W algorytmie tym mamy do czynienia z hiperparametrami - jest to między innymi rozmiar populacji, prawdopodobieństwo krzyżowania czy prawdopodobieństwo mutacji. Również, w ramach algorytmów genetycznych, konieczne jest zdefiniowanie funkcji dopasowania pasującej do problemu, metody selekcji nowej populacji, czy sposobie krzyżowania i mutowania dzieci.

\subsection{Zastosowanie algorytmu do Sudoku}

Na początku zdefiniujmy sam chromosom - będzie to ciąg 81 liczb z zakresu 1-9, podzielonych na 9 podciągów, reprezentujących jeden podblok sudoku 3x3. Do algorytmu wprowadzony jest również ciąg pomocniczy - definiuje on stałe, niezmienne liczby w chromosomie - składa się on zarówno z liczb z zakresu 1-9 definiujących stałe miejsca w planszy oraz liczb 0 - oznaczających miejsce, które może być w wyniku kolejnych generacji zmieniane w chromosomie.\\

Kolejnym elementem do zdefiniowania jest warunek stopu - zakładamy tutaj, że generowanie kolejnych populacji jest zakończone dopiero, gdy całe sudoku zostanie rozwiązane. Ze względu na możliwość utykania przez sudoku w optimach lokalnych, dobrym pomysłem może okazać się zastosowanie dodatkowego elementu w algorytmie - jeśli liczba iteracji w ramach których się nie poprawił wynik będzie zbyt duża, można przeprowadzić reinicjalizację populacji poprzez zachowanie małej części obecnej populacji i dopełnienie jej o nowe, losowo wygenerowane chromosomy.\\

Sudoku swoim charakterem przypomina problem optymalizacji ze spełnianiem ograniczeń - aby rozwiązanie było poprawne musi ono spełniać następujące warunki:
\begin{itemize}
  \item ciąg elementów w każdej kolumnie sudoku musi być permutacją ciągu [1 2 3 4 5 6 7 8 9],
  \item ciąg elementów w każdym wierszu sudoku musi być permutacją ciągu [1 2 3 4 5 6 7 8 9],
  \item ciąg elementów w każdym podbloku 3x3 musi być permutacją ciągu [1 2 3 4 5 6 7 8 9],
  \item w wynikowej planszy nie może nastąpić zmiana liczby w miejscach, które zostały podane na wejściu.
\end{itemize}
Zauważmy, jednak, że jeśli odpowiednio zajmiemy się losowaniem populacji startowej, mutacją i krzyżowaniem, warunek z ciągiem elementów w każdym podbloku będzie zawsze spełniany. Dodatkowo, jeśli wprowadzimy podany wcześniej ciąg pomocniczy, ostatni z tych warunków też zapewnimy - w ten sposób dla funkcji celu mamy do spełnienia 18 ograniczeń (osobno na każdy wiersz i osobno na każdą kolumnę).\\

Losowanie populacji startowej będzie odbywało się oddzielnie w ramach każdego podciągu każdego chromosomu - w wolnych miejscach w podciągu losowo rozmieścimy wszystkie pozostałe do rozłożenia liczby. Wtedy mamy zapewnione, że w populacji startowej, w każdym chromosomie nie ma naruszenia warunku z podblokiem 3x3.\\

Selekcja jaką zastosujemy do zachowania części populacji to selekcja elitarna ze współczynnikiem elitarności wynoszącym 2 (co oznacza, że 2 najlepsze chromosomy z poprzedniej populacji pozostaną niezmienione w następnej), a wybór rodziców do krzyżowania będzie wykonywany przy użyciu selekcji turniejowej.\\

Krzyżowanie jakie zostanie zastosowane to krzyżowanie jednorodne, będzie wykonywane tylko mię\-dzy podblokami, tak więc dla ciągu 81 elementów możliwych punktów krzyżujących jest 8.\\

Mutacje, wykonywane z odpowiednim prawdopodobieństwem, będą miały miejsce tylko wewnątrz danego podbloku (w ramach podciągu wielkości 9 elementów), zastosujemy w tym przypadku mutację wymieniającą (zastępującą liczby miejscami), jednak gdy wymiana naruszy pozycje liczb w stosunku do ciągu wejściowego, mutacja będzie porzucana. \\

Po pojedynczym krzyżowaniu i zastosowaniu mutacji na wynikach krzyżowania otrzymujemy dwa nowe chromosomy, będące częścią nowej populacji. Warto zwrócić uwagę na fakt, że te dwie operacje, zastosowane w powyższy sposób, nie naruszają zarówno warunku z prawidłowym ciągiem elementów w podbloku 3x3 Sudoku, jak i nie zmieniają liczb nienaruszalnych z punktu widzenia planszy wejściowej. \\

Ostatnim elementem pozostałym do zdefiniowania jest funkcja dopasowania $f$, którą chcemy minimalizować. Optymalne rozwiązanie (pełne rozwiązanie Sudoku) oznacza znalezienie chromosomu, dla którego $f=0$. Funkcja, jaką proponujemy, polega na zainicjowaniu $f\gets0$ i przejściu po wszystkich wierszach i kolumnach Sudoku i dodaniu kary w postaci ilości nieznajdujących się w ramach wiersza lub kolumny liczb ze zbioru $\{1,2,3,4,5,6,7,8,9\}$. Dodatkowym pomysłem jest dodanie do wartości funkcji $f\gets f+i$, gdy najlepszy wynik z aktualnej populacji jest nie lepszy niż najlepszy z poprzednich $i$ epok, co ułatwić może uciekanie od optimów lokalnych.\\

\section{Testy numeryczne}
\begin{itemize}
\item Porównanie efektywności algorytmu ACO dla różnej liczby mrówek.
\item Porównanie efektywności algorytmów dla plansz Sudoku o różnym poziomie trudności (przy czym trudność plansz oceniana będzie na podstawie literatury).
\item Porównanie efektywności algorytmów dla różnych wartości hiperparametrów, takich jak prawdopodobieństwa mutacji i krzyżowania czy rozmiar populacji.
\item Porównanie czasu potrzebnego algorytmom do znalezienia rozwiązania.
\item Porównanie liczby epok i wywołań funkcji dopasowania potrzebnych do znalezienia rozwiązania w algorytmie genetycznym.
\item Przy testowaniu efektywności algorytmów będą one uruchamiane kilkanaście razy - będziemy mierzyć średnią, medianę, odchylenie standardowe wyników.
\end{itemize}

\section{Wybrane technologie}
Algorytmy zaimplementowane zostaną w języku Python. Do wizualizacji wyników i porównywania efektywności algorytmów wykorzystamy bibliotekę Matplotlib. Klasy potrzebne do implementacji wykonane zostaną jako skrypty Python'owe, natomiast wizualizacja, porównywanie i testowanie algorytmów wykonamy w Jupyter Notebook'ach - dla czytelności odczytu i wygody wykonywania wielu testów przy jednoczesnej ich wizualizacji na wykresach.

\section{Bibliografia}
[1] Mirjalili, S. (2019). "Evolutionary Algorithms and Neural Networks: Theory and Applications". Griffith University Brisbane, QLD: Springer International Publishing.

[2] Lloyd, H. and Amos, M. (2020). "Solving Sudoku With Ant Colony Optimization". IEEE.

[3] Dorigo, M., Birattari, M. and Stutzle, T. “Ant Colony Optimization” J. Autom., IEEE, 2006.

[4] T. Mantere and J. Koljonen, "Solving, rating and generating Sudoku puzzles with GA," 2007 IEEE Congress on Evolutionary Computation, 2007, pp. 1382-1389, doi: 10.1109/CEC.2007.4424632.

\end{document}
