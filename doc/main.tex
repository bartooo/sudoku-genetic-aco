%%%%%%%%%%%%%%%%%%%%%%%%%%%%%%%%%%%%%%%%%
% Wenneker Assignment
% LaTeX Template
% Version 2.0 (12/1/2019)
%
% This template originates from:
% http://www.LaTeXTemplates.com
%
% Authors:
% Vel (vel@LaTeXTemplates.com)
% Frits Wenneker
%
% License:
% CC BY-NC-SA 3.0 (http://creativecommons.org/licenses/by-nc-sa/3.0/)
% 
%%%%%%%%%%%%%%%%%%%%%%%%%%%%%%%%%%%%%%%%%

%----------------------------------------------------------------------------------------
%	PACKAGES AND OTHER DOCUMENT CONFIGURATIONS
%----------------------------------------------------------------------------------------

\documentclass[11pt]{scrartcl} % Font size

%%%%%%%%%%%%%%%%%%%%%%%%%%%%%%%%%%%%%%%%%
% Wenneker Assignment
% Structure Specification File
% Version 2.0 (12/1/2019)
%
% This template originates from:
% http://www.LaTeXTemplates.com
%
% Authors:
% Vel (vel@LaTeXTemplates.com)
% Frits Wenneker
%
% License:
% CC BY-NC-SA 3.0 (http://creativecommons.org/licenses/by-nc-sa/3.0/)
% 
%%%%%%%%%%%%%%%%%%%%%%%%%%%%%%%%%%%%%%%%%

%----------------------------------------------------------------------------------------
%	PACKAGES AND OTHER DOCUMENT CONFIGURATIONS
%----------------------------------------------------------------------------------------

\usepackage{amsmath, amsfonts, amsthm} % Math packages
\usepackage{algorithm}
\usepackage{algpseudocode}
\usepackage{listings} % Code listings, with syntax highlighting

\usepackage[english]{babel} % English language hyphenation
\usepackage{amsmath}
\DeclareMathOperator*{\argmax}{arg\,max}
\usepackage{graphicx} % Required for inserting images
\graphicspath{{Figures/}{./}} % Specifies where to look for included images (trailing slash required)

\usepackage{booktabs} % Required for better horizontal rules in tables

\numberwithin{equation}{section} % Number equations within sections (i.e. 1.1, 1.2, 2.1, 2.2 instead of 1, 2, 3, 4)
\numberwithin{figure}{section} % Number figures within sections (i.e. 1.1, 1.2, 2.1, 2.2 instead of 1, 2, 3, 4)
\numberwithin{table}{section} % Number tables within sections (i.e. 1.1, 1.2, 2.1, 2.2 instead of 1, 2, 3, 4)

\setlength\parindent{0pt} % Removes all indentation from paragraphs

\usepackage{enumitem} % Required for list customisation
\setlist{noitemsep} % No spacing between list items

%----------------------------------------------------------------------------------------
%	DOCUMENT MARGINS
%----------------------------------------------------------------------------------------

\usepackage{geometry} % Required for adjusting page dimensions and margins

\geometry{
	paper=a4paper, % Paper size, change to letterpaper for US letter size
	top=2.5cm, % Top margin
	bottom=3cm, % Bottom margin
	left=3cm, % Left margin
	right=3cm, % Right margin
	headheight=0.75cm, % Header height
	footskip=1.5cm, % Space from the bottom margin to the baseline of the footer
	headsep=0.75cm, % Space from the top margin to the baseline of the header
	%showframe, % Uncomment to show how the type block is set on the page
}

%----------------------------------------------------------------------------------------
%	FONTS
%----------------------------------------------------------------------------------------

\usepackage[utf8]{inputenc} % Required for inputting international characters
\usepackage[T1]{fontenc} % Use 8-bit encoding

\usepackage{fourier} % Use the Adobe Utopia font for the document

%----------------------------------------------------------------------------------------
%	SECTION TITLES
%----------------------------------------------------------------------------------------

\usepackage{sectsty} % Allows customising section commands

\sectionfont{\vspace{6pt}\centering\normalfont\scshape} % \section{} styling
\subsectionfont{\normalfont\bfseries} % \subsection{} styling
\subsubsectionfont{\normalfont\itshape} % \subsubsection{} styling
\paragraphfont{\normalfont\scshape} % \paragraph{} styling

%----------------------------------------------------------------------------------------
%	HEADERS AND FOOTERS
%----------------------------------------------------------------------------------------

\usepackage{scrlayer-scrpage} % Required for customising headers and footers

\ohead*{} % Right header
\ihead*{} % Left header
\chead*{} % Centre header

\ofoot*{} % Right footer
\ifoot*{} % Left footer
\cfoot*{\pagemark} % Centre footer
 % Include the file specifying the document structure and custom commands

%----------------------------------------------------------------------------------------
%	TITLE SECTION
%----------------------------------------------------------------------------------------

\title{	
	\normalfont\normalsize
	\rule{\linewidth}{0.5pt}\\ % Thin top horizontal rule
	\vspace{20pt} % Whitespace
	{\huge Rozwiązywanie Sudoku przy pomocy algorytmu genetycznego oraz algorytmu ACO}\\ % The assignment title
	\vspace{12pt} % Whitespace
	\rule{\linewidth}{2pt}\\ % Thick bottom horizontal rule
}

\author{\LARGE Bartosz Cywiński, Łukasz Staniszewski} % Your name

\date{} % Today's date (\today) or a custom date

\begin{document}

\maketitle % Print the title
%----------------------------------------------------------------------------------------
%	FIGURE EXAMPLE
%----------------------------------------------------------------------------------------

\section{Opis problemu}
Podstawowa wersja Sudoku składa się z dwuwymiarowej planszy 9x9, podzielonej na 9 pól 3x3. Planszę Sudoku da się jednak rozszerzać do większych rozmiarów. Plansza na początku gry częściowo uzupełniona jest liczbami. Początkowy rozkład liczb ma duże znaczenie dla dalszego przebiegu gry – jest on jedynym czynnikiem decydującym o trudności danej wersji Sudoku. Celem rozwiązania łamigłówki jest takie ułożenie cyfr od 1 do 9, aby w każdym rzędzie, każdej kolumnie i każdym polu 3x3 każda cyfra występowała dokładnie jeden raz. Zostało udowodnione, że problem rozwiązania Sudoku jest problemem NP-trudnym, dlatego też jest on dobrym testem efektywności algorytmów.

%------------------------------------------------
\section{Opis rozwiązania przy pomocy algorytmu ACO}


\subsection{Inspiracja algorytmu}

Główną inspiracją algorytmu jest koncept stygmergii w naturze – manipulacji środowiskiem przez organizmy w celu komunikowania się ze sobą. Mrówki ciągle poszukując pożywienia poruszają się w losowych kierunkach. Kiedy znajdą drogę do pożywienia, oznaczają ścieżkę feromonami – przy czym ilość feromonu zależy od ilości, jakości pożywienia oraz długości ścieżki. Kiedy inne mrówki wyczują obecność feromonu, podążają za nim, dzięki czemu docierają do pożywienia.

\subsection{Pseudokod i opis algorytmu}
\begin{algorithm}
\caption{Algorytm ACS do rozwiązywania Sudoku}\label{alg:cap}
\begin{algorithmic}[1]
\State \texttt{wczytaj planszę Sudoku}
\For{\texttt{komórka z wybraną wartością}}
	\State \texttt{aktualizuj możliwe do przyjęcia wartości przez sąsiadów tej komórki}
\EndFor
\State \texttt{zainicjuj tablicę feromonów}
\While {\texttt{!stop}}
	\State \texttt{daj każdej mrówce kopię planszy Sudoku}
	\For{\texttt{liczba komórek}}
		\For{\texttt{liczba mrówek}}
			\If{\texttt{komórka nie ma wybranej wartości}}	
				\State \texttt{wybierz wartość dla komórki}
				\State \texttt{aktualizuj ograniczenia}
				\State \texttt{aktualizuj lokalny feromon}
			\EndIf
		\EndFor
	\EndFor
	\State \texttt{znajdź najlepszą mrówkę}
	\State \texttt{aktualizuj tablicę feromonów}
	\State \texttt{parowanie najlepszej wartości}
\EndWhile

\end{algorithmic}
\end{algorithm}

W zaproponowanym rozwiązaniu użyty został algorytm Ant Colony System (ACS) - jest to ulepszona wersja klasycznego algorytmu Ant System (AS). Jego charakterystyczną cechą jest wprowadzenie lokalnych aktualizacji feromonów przez każdą mrówkę oraz to, że tylko najlepsza mrówka aktualizuje globalne wartości tablicy feromonów. Ponadto po każdym wybraniu nowego elementu rozwiązania (wpisaniu cyfry w wybraną komórkę planszy) aktualizowane zostają wszystkie wartości możliwe do przyjęcia przez komórki planszy spełniające zasady łamigłówki, ponieważ na skutek wybrania elementu rozwiązania mogły się one zmienić.\\


\textit{Linie 2-4:} Każda komórka o indeksie $i$ na początku algorytmu posiada taki sam zbiór możliwych wartości do przyjęcia ($\mathbb{V}_i=\{1,...,9\}$). Aktualizacje ograniczeń komórek z wybraną wartością polegają na:
\begin{itemize}
	\item Eliminacji z $\mathbb{V}_i$ wartości wybranych przez sąsiadów komórki $i$, gdzie sąsiedzi to wszystkie komórki w tym samym rzędzie, tej samej kolumnie oraz tym samym polu 3x3, 
	\item Jeśli $|\mathbb{V}_i| = 1$, to przypisz wartość do komórki. 
\end{itemize}


\textit{Linia 5:} Dla Sudoku o wymiarze $d$ definiujemy dwuwymiarową globalną tablicę feromonów $\tau$, w której każdy element jest oznaczony jako $\tau_{i}^{k}$, gdzie $i$ to indeks komórki planszy $(0\leq{i}\leq{d^2-1})$, a $k$ to możliwa wartość do przyjęcia przez komórkę $(k\in{[1,d]})$. Każdy element tablicy feromonów inicjowany jest wartością $\tau_{0}=\frac{1}{c}$, gdzie $c=d^2$ jest całkowitą liczbą komórek planszy. W tablicy trzymane są wartości feromonów każdej możliwej do przyjęcia wartości w każdej komórce planszy.\\

\textit{Linie 8-10:} Zbiór $m$ mrówek buduje częściowe rozwiązanie $s^p$ ze skończonego zbioru możliwych elementów rozwiązania $\textbf{C} = \{c_{i}^{k}\}$, $i=0,...,d-1,$ $k\in{[1,d]}$, gdzie $i$ to indeks komórki planszy, a $k$ to możliwa cyfra do wpisania w komórkę. Każda mrówka zaczyna od losowo wybranej komórki planszy Sudoku. Przechodzi ona po każdej możliwej komórce na planszy. Za każdym razem, kiedy mrówka natrafia na komórkę, która nie ma przypisanej wartości, podejmuje ona decyzję jaka to powinna być wartość i ją przypisuje - do $s^p$ dodawany jest element rozwiązania ze zbioru $\textbf{N}(s^p)\subseteq{\textbf{C}}$, który jest zdefiniowany jako zbiór elementów, które mogą być dodane do częściowego rozwiązania $s^p$ bez naruszenia założonych ograniczeń - w przypadku Sudoku każda cyfra może występować tylko jeden raz w każdym wierszu, kolumnie oraz w polach 3x3.\\

\textit{Linia 11:} Wybór wartości dla komórki $i$ dokonywany jest za pomocą stochastycznego mechanizmu uwzględniając wartości w tablicy feromonów. Zdefiniujmy zbiór $\mathbb{V}_i$, jako zbiór możliwych do przyjęcia wartości przez komórkę $i$. Istnieją do wyboru dwie metody:
\begin{itemize}
	\item Selekcja zachłanna, gdzie wybieramy element ze zbioru $\mathbb{V}_i$ z najwyższą wartością w tablicy feromonów, 
	\item Selekcja ruletkowa, gdzie prawdopodobieństwa są proporcjonalne do wartości w tablicy feromonów. 
\end{itemize}
Prawdopodobieństwa wybierane są na podstawie parametru $ q_0\in{[0,1]} $. Kolejny element rozwiązania $s$ jest zatem wybierany według wzoru:
\begin{equation}
s = \begin{cases}
\argmax_{{k\in\mathbb{V}_i}}{\tau_i^k}, & \text{jeśli $q\leq{q_0}$},\\
$R$, & \text{w pozostałych przypadkach},
\end{cases}
\end{equation}
gdzie wartość $ q \in{[0,1]}$ losowana jest z rozkładem jednostajnym, natomiast $R$ wybierana jest na podstawie:
\begin{equation}
p_{i}^{k} =
\dfrac{\tau_{i}^{k}}{\sum_{j\in{\mathbb{V}_i}}{\tau_i^j}}, k\in\mathbb{V}_i
\end{equation}
gdzie $p_{i}^{k}$ to prawdopodobieństwo wyboru $k$ z $\mathbb{V}_i$ na miejscu o indeksie $i$.\\

\textit{Linia 13:} Aktualizowanie lokalnego feromonu wspomaga eksplorację przestrzeni przeszukiwań. Za każdym razem kiedy wartość $s$ zostaje przypisana do komórki w planszy o indeksie $i$, aktualizacja następuje według formuły:
\begin{equation}
\tau_i^s = (1 - \varphi)\tau_i^s + \varphi\tau_0,
\end{equation}
gdzie $\varphi \in (0,1]$ to parametr rozpadu feromonu, a $\tau_0$ jest początkową wartością feromonu.\\


Celem aktualizacji tablicy feromonów jest zwiększenie wartości feromonów związanych z dobrymi rozwiązaniami i zmniejszenie wartości związanych z gorszymi rozwiązaniami. Jest to osiągane przez:
\begin{itemize}
	\item Zmniejszenie wszystkich wartości feromonów przez \textit{parowanie}, 
	\item Zwiększenie wartości feromonów związanych z dobrymi rozwiązaniami.
\end{itemize}

\textit{Linie 17-18:} Aby znaleźć najlepsze rozwiązanie, należy znaleźć najlepszą mrówkę w iteracji, czyli taką, która uzupełniła najwięcej komórek w planszy Sudoku w danej iteracji algorytmu. Aktualizacja tablicy feromonów następuje według wzoru:
\begin{equation}
\tau_i^s = \begin{cases}
(1-\rho)\tau_i^s+\rho\Delta\tau_i^s, & \text{jeśli najlepsze rozwiązanie w iteracji jest najlepsze globalnie},\\
\tau_i^s, & \text{w pozostałych przypadkach,}
\end{cases}
\end{equation}
gdzie $\rho \in [0,1]$ to parametr parowania feromonu, a $\Delta\tau_i^s$ można przyjąć jako:
\begin{equation}
\Delta\tau_i^s = \dfrac{c}{c-f_{best}},
\end{equation}
gdzie $f_{best}$ to liczba komórek uzupełnionych przez najlepszą mrówkę.\\

\textit{Linia 19:} Na koniec stosowane jest parowanie najlepszej wartości, które zapobiega przedwczesnej zbieżności algorytmu:
\begin{equation}
\Delta\tau_{best} = \Delta\tau_{best}(1-\rho_{best}),
\end{equation}
gdzie $\tau_{best}$ to parametr parowania feromonu najlepszej wartości.\\

Po spełnieniu warunku stopu, a więc po zupełnym rozwiązaniu Sudoku lub po osiągnięciu limitu iteracji, zwracane jest rozwiązanie w postaci przynajmniej częściowo uzupełnionej planszy Sudoku.

\section{Bibliografia}
[1] Mirjalili, S. (2019). "Evolutionary Algorithms and Neural Networks: Theory and Applications". Griffith University Brisbane, QLD: Springer International Publishing.

[2] Lloyd, H. and Amos, M. (2020). "Solving Sudoku With Ant Colony Optimization". IEEE.

[3] Dorigo, M., Birattari, M. and Stutzle, T. “Ant Colony Optimization” J. Autom., IEEE, 2006.
\end{document}
